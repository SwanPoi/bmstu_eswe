\section{Конструкторский раздел}

\subsection{Последовательность действий}

На рисунках~\ref{fig:init}~---~\ref{fig:exit} приведена последовательность действий при загрузке и выгрузке модуля.

\begin{figure}[H]
	\centering
	\includesvg[width=0.9\textwidth]{img/idefload.svg}
	\caption{Последовательность действий при загрузке модуля}
	\label{fig:init}
\end{figure}

\begin{figure}[H]
	\centering
	\includesvg[width=0.9\textwidth]{img/idefunload.svg}
	\caption{Последовательность действий при выгрузке модуля}
	\label{fig:exit}
\end{figure}

\subsection{Разработка алгоритма работы предобработчика}

На рисунке~\ref{fig:pre_hook} приведена схема действий, выполняемых предобработчиком функции $task\_tick\_fair$.

\begin{figure}[H]
	\centering
	\includesvg[width=0.9\textwidth]{img/pre_hook.svg}
	\caption{Схема алгоритма работы предобработчика функции task\_tick\_fair}
	\label{fig:pre_hook}
\end{figure}

В предобработчике необходимо получить второй аргумент функции \\ $task\_tick\_fair$, содержащий указатель на $\text{struct task\_struct}$.
Если указатель не NULL и поле $pid$ совпадает с идентификатором искомого процесса, необходимо добавить информацию из структуры $task\_struct$ в список информации о параметрах планирования.

\subsection{Разработка алгоритма создания элемента списка}

На рисунке~\ref{fig:add} приведена схема создания элемента списка.

\begin{figure}[H]
	\centering
	\includesvg[width=0.9\textwidth]{img/add_new.svg}
	\caption{Схема создания элемента списка}
	\label{fig:add}
\end{figure}

В случае успешного выделения памяти под элемент списка выделяется память под поля, хранящие информацию о параметрах планирования для Real-Time и Deadline планировщиков.
Если при выделении памяти произошла ошибка, то память, выделенная под элемент списка, освобождается, а в качестве возвращаемого значения используется NULL.

\subsection{Структура программного обеспечения}

Разрабатываемое ПО должно быть реализовано в виде загружаемого модуля ядра, в качестве параметра которого указывается идентификатор отслеживаемого процесса.
Для передачи пользователю информации о планировании процесса доkжен использоваться sequence-файл в $proc$.
Структура программного обеспечения представлена на рисунке~\ref{fig:program_structure}.
\begin{figure}[H]
	\centering
	\includesvg[width=0.9\textwidth]{img/program.svg}
	\caption{Структура программного обеспечения}
	\label{fig:program_structure}
\end{figure}

