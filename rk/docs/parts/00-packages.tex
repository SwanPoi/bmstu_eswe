\usepackage[14pt]{extsizes}

\usepackage{cmap} % Улучшенный поиск русских слов в полученном pdf-файле
\usepackage[T2A]{fontenc} % Поддержка русских букв
\usepackage[utf8]{inputenc} % Кодировка utf8
\usepackage[english,russian]{babel} % Языки: английский, русский
\usefont{T2A}{ftm}{m}{sl} % Основная строчка, которая позволяет получить шрифт Times New Roman

\usepackage[left=30mm,right=10mm,top=20mm,bottom=20mm]{geometry}

\usepackage{pdfpages}

\usepackage{enumitem}
\usepackage{multirow}
\usepackage[para,online,flushleft]{threeparttable}
\usepackage{caption}
% Работа с изображениями и таблицами; переопределение названий по ГОСТу
\captionsetup[table]{singlelinecheck=false, labelsep=endash}
\captionsetup[table]{justification=raggedright,singlelinecheck=off}
\captionsetup{labelsep=endash}
\captionsetup[figure]{name={Рисунок}}
\captionsetup[lstlisting]{justification=raggedright,singlelinecheck=off} % Работа с листингом
\usepackage[justification=centering]{caption} % Настройка подписей float объектов

\usepackage{graphicx}
\usepackage{slashbox}
\usepackage{diagbox} % Диагональное разделение первой ячейки в таблицах

\usepackage{amssymb}
\usepackage{amsmath}
\usepackage{float}
\usepackage{csvsimple}
\usepackage{enumitem} 
%\setenumerate[0]{label=\arabic.)} % Изменение вида нумерации списков
\renewcommand{\labelitemi}{---}

% Переопределение стандартных \section, \subsection, \subsubsection по ГОСТу;
% Переопределение их отступов до и после для 1.5 интервала во всем документе
\usepackage{titlesec}

\titleformat{\section}[block]
{\bfseries\normalsize\filcenter}{\thesection}{1em}{}

\titleformat{\subsection}[hang]
{\bfseries\normalsize}{\thesubsection}{1em}{}
\titlespacing\subsection{\parindent}{12mm}{12mm}

\titleformat{\subsubsection}[hang]
{\bfseries\normalsize}{\thesubsubsection}{1em}{}
\titlespacing\subsubsection{\parindent}{12mm}{12mm}

% Список литературы
\makeatletter 
\def\@biblabel#1{#1 } % Изменение нумерации списка использованных источников
\makeatother

\usepackage{setspace}
\onehalfspacing % Полуторный интервал

\frenchspacing
\usepackage{indentfirst} % Красная строка после заголовка
\setlength\parindent{1.25cm}
\usepackage{multirow}
\renewcommand{\baselinestretch}{1.5}
\def\arraystretch{1.5}%  1 is the default, change whatever you need
\setlist{nolistsep} % Отсутствие отступов между элементами \enumerate и \itemize

% Цвета для гиперссылок и листингов
\usepackage{xcolor}
\usepackage{color} 

\renewcommand*{\arraystretch}{1} % математические матрицы интервал между строками

\usepackage{ulem} % Нормальное нижнее подчеркивание
\usepackage{hhline} % Двойная горизонтальная линия в таблицах
\usepackage[figure,table]{totalcount} % Подсчет изображений, таблиц
\usepackage{rotating} % Поворот изображения вместе с названием
\usepackage{lastpage} % Для подсчета числа страниц

% Дополнительное окружения для подписей
\usepackage{array}
\newenvironment{signstabular}[1][1]{
	\renewcommand*{\arraystretch}{#1}
	\tabular
}{
	\endtabular
}

\usepackage{pgfplots}
\usepackage{pgfplotstable}
\pgfplotsset{compat = newest}
\usetikzlibrary{datavisualization}
\usetikzlibrary{datavisualization.formats.functions}

\usepackage[justification=centering]{caption} % Настройка подписей float объектов

\usepackage[unicode,pdftex]{hyperref} % Ссылки в pdf
\hypersetup{hidelinks}

\newcommand{\code}[1]{\texttt{#1}}

% используется в качетсве обозначения в таблицах сравнения
\usepackage{pifont}
\newcommand{\cmark}{{\ding{51}}}%
\newcommand{\xmark}{{\ding{55}}}%

\usepackage{listings}
% Для листинга кода:
\lstset{%
	language=sql,   					% выбор языка для подсветки	
	basicstyle=\small\sffamily,			% размер и начертание шрифта для подсветки кода
	numbersep=5pt,
	numbers=left,						% где поставить нумерацию строк (слева\справа)
	%numberstyle=,					    % размер шрифта для номеров строк
	stepnumber=1,						% размер шага между двумя номерами строк
	xleftmargin=17pt,
	showstringspaces=false,
	numbersep=5pt,						% как далеко отстоят номера строк от подсвечиваемого кода
	frame=single,						% рисовать рамку вокруг кода
	tabsize=4,							% размер табуляции по умолчанию равен 4 пробелам
	captionpos=t,						% позиция заголовка вверху [t] или внизу [b]
	breaklines=true,					
	breakatwhitespace=true,				% переносить строки только если есть пробел
	escapeinside={\#*}{*)},				% если нужно добавить комментарии в коде
	backgroundcolor=\color{white}
}

\usepackage{verbatim}
\usepackage{tabularx}
\usepackage{svg}

\usepackage{algorithm}
\makeatletter
\renewcommand{\ALG@name}{Листинг}
\makeatother
\usepackage{algpseudocode}

% Настройка абзацного отступа
\RequirePackage{indentfirst}
\setlength{\parindent}{12.5mm}
% Настройка полей
\RequirePackage[
left=30mm,
right=10mm, % Является требованием МГТУ, не соответствует ГОСТ 7.32-2017
top=20mm,
bottom=20mm,
]{geometry}