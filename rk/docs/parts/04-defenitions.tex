%\specsection{ОПРЕДЕЛЕНИЯ}
\section*{\centering ОПРЕДЕЛЕНИЯ}
\addcontentsline{toc}{section}{ОПРЕДЕЛЕНИЯ}

Тангаж --- угловое движение объекта, при котором его продольная ось изменяет свое направление относительно горизонтальной плоскости~\cite{palcing-camera}.

Рыскание --- угловые движения объекта вокруг вертикальной оси~\cite{palcing-camera}.

Graphical user interface --- <<Графический интерфейс пользователя>> обеспечивает возможность управления поведением вычислительной системы через визуальные элементы управления — окна, списки, кнопки, гиперссылки и т.д.~\cite{termin-cg}

Буфер кадра --- часть графической памяти для хранения массива кодов, определяющих засветку пикселей на экране~\cite{termin-cg}.

Визуализация (англ. Rendering) --- создание плоских изображений трехмерных (3D) моделей~\cite{termin-cg}.

Перспективная (центральная) проекция --- вид проекции, где лучи проектирования исходят из одного центра (центра проектирования), размещенного на конечном расстоянии от объектов и плоскости проектирования~\cite{termin-cg}.
