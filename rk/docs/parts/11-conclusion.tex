\phantomsection
\section*{\centering ЗАКЛЮЧЕНИЕ}
\addcontentsline{toc}{section}{ЗАКЛЮЧЕНИЕ}

В результате выполнения курсовой работы проведен анализа структуры дескриптора процесса и выделены поля, связанные с планированием: $prio$, $rt\_priority$, $normal\_prio$, $static\_prio$, $stats$, $sched\_info$, $policy$, $se$, $rt$, $dl$, $sched\_class$.
Проанализированы следующие структуры: $sched\_statistics$, $sched\_info$, $rq$, $cfs\_rq$, $srt\_rq$, $dl\_rq$,  $sched\_entity$, $sched\_rt\_entity$, \\ $sched\_dl\_entity$.

Проведен анализ структуры $sched\_class$ и алгоритмов планирования: CFS, Deadline, Real-Time.
Определены поля, влияющие на работу рассмотренных алгоритмов:
\begin{itemize}[label=---]
	\item для планировщика CFS~---~поле $vruntime$, находящееся в структуре $sched\_entity$;
	\item для Deadline планировщика~---~поля $dl\_runtime$, $dl\_deadline$, $dl\_period$, $runtime$, $deadline$, хранящиеся в структуре $sched\_dl\_entity$;
	\item для Real-Time планировщика~---~поле $time\_slice$ из структуры \\ $sched\_rt\_entity$.
\end{itemize}

Проведен сравнительный анализ способов перехвата функций ядра, в результате которого выбраны kprobes и ftrace, так как они не требуют перекомпиляции ядра и поддерживаются современными версиями ядра.

В рамках курсовой работы были выполнены поставленные задачи:
\begin{itemize}[label=---]
	\item проведен анализ структур и функций, предоставляющих возможность реализовать поставленную задачу;
	\item проведен анализ способов перехвата функций;
	\item разработаны алгоритмы и структура загружаемого модуля, обеспечивающего отслеживание работы планировщика;
	\item спроектирован и реализован загружаемый модуль ядра;
	\item проанализирована работа загружаемого модуля ядра.
\end{itemize}

Исследование разработанного программного обеспечения показало, что оно соответствует техническому заданию и выполняет все поставленные задачи.


