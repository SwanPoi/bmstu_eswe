\section{Исследовательский раздел}

Программное обеспечение реализовано на дистрибутиве KUBUNTU с ядром версии 6.8.0.

\subsection{Демонстрация работы программы}

Для получения результатов для процесса реального времени использовалось воспроизведение музыки в браузере Mozilla Firefox.
На рисунке~\ref{fig:rt_example} приведена таблица параметров планирования для процесса реального времени.

\begin{figure}[H]
	\centering
	\includegraphics[width=0.9\textwidth]{img/rt_example.png}
	\caption{Демонстрация таблицы параметров планирования для процесса реального времени}
	\label{fig:rt_example}
\end{figure}

Для получения результатов для процесса с дисциплиной планирования SCHED\_NORMAL была использована программа, получающая на вход размеры двух матриц и их элементы и выводящая результат умножения.
Ни рисунке~\ref{fig:cfs_example} приведена таблица параметров планирования для процесса с дисциплиной планирования SCHED\_NORMAL.

\begin{figure}[H]
	\centering
	\includegraphics[width=0.9\textwidth]{img/cfs_example.png}
	\caption{Демонстрация таблицы параметров планирования для процесса с дисциплиной планирования SCHED\_NORMAL}
	\label{fig:cfs_example}
\end{figure}

